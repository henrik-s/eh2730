\section{Introduction}
\label{sec:introduction}
This document is a Work Order Management System requirement specification intended for a procurement project of the ACME company. This section describes the project's purpose and background, including the business goals. A \emph{Document Overview} and a \emph{Glossary} can be also be found under this topic.
\subsection{Purpose and Background}
\label{purpose_and_background}
This document is a project plan for the procurement project of "The ACME WOMS project". "The ACME WOMS project" will, if executed, implement a new digital system in the current ACME system collection. This system will be a Work Order Management System (WOMS) targeted to optimize proactive/reactive maintenance and the customer support process, in terms of costs and efficiency.

The background for this procurement project is that the company, ACME, has an outstanding economic issue, which has been a problem the company for a long time \cite{A}. The company has located a low profitability, which they think is an indicator that there may be sub-optimal solutions within the company. ACME has also analyzed the costs of the company and located that one of the biggest cost expenses are the customer support team. Furthermore, recently a new legislation was taken in action enacting a penalty fee for outages longer than 12 hours, increasing the pressure to an already strained budget. The management department believes that cost reductions can be made in the maintenance and repair section if a better control system would be present.

\subsubsection{Vision Statement}
\label{vision_statement}
For ACME who deliver electrical power to factories and households, where the main costs are maintenance and customer support, the WOMS, a COTS software integrated with other systems in the organization, will enhance how proactive and reactive maintenance is functioning and how information can be provided to customers.

Unlike existing work procedures that do not offer the ability to control neither how to dispatch work orders nor to follow up and reduce costs related to maintenance work, the WOMS product will fulfill not only how to control and optimize the maintenance function, but also enable ACME's customer support to provide better and more up-to-date information to customers. 

\subsection{Goals}
The main costs of ACME's business are maintenance, repair and construction and the survey indicates those costs to be beyond industry average \cite{A}. The major goal of this project is to resolve this issue, lower the costs to industry average, increasing the competitive value of the ACME Company. Within the same scope, another goal is to tackle the "12 hour outage penalty" issue by cutting the costs for this post in half.

Another major cost can be found in the customer care division. A WOMS \cite{B} can support this process, providing function and service to improvement and optimization. A concrete example: it will set the preconditions needed to automate this process, e.g. having an ''computerized'' customer care function providing customer with outage information. The overall intention and goals for the customer care division, in this project, is to decrease the costs for the customer care team division.

All the goals is summarized in the table \ref{table:goals}. The goals will be compared to economic report for the time period January 1st 2012 to December 31st 2012.
\begin{center}
	\begin{longtable}{|l|p{3.5cm}|p{3.5cm}|l|}
		\caption{Goals}
		\label{table:goals}\\
		\hline
		\textbf{ID} & \textbf{Description} & \textbf{How} & \textbf{Due Date} \\
		\hline
		\endfirsthead

		\multicolumn{4}{c}%
		{\tablename\ \thetable\ -- \textit{Continued from previous page}} \\
		\hline
		\textbf{ID} & \textbf{Description} & \textbf{How} & \textbf{Due Date} \\
		\hline
		\endhead

		\hline \multicolumn{4}{r}{\textit{Continued on next page}} \\
		\endfoot

		\hline
		\endlastfoot
		%
		GOAL-1 \label{goal-1} & 
		Reduce the expenses for the ''maintenance, repair, and construction'' post to < 100\% of industry average &
		By the support of WOMS functionality, improve proactive and reactive maintenance processes &
		May 1st 2016 \\
		\hline
		%
		GOAL-2 \label{goal-2}&
		Lower the costs for "12 hour outage penalty" cases by 50\% &
		Embracing prioritized and efficient dispatching of work orders &
		May 1st 2015 \\
		\hline
		%
		GOAL-3 \label{goal-3}&
		A 30\% decrease of costs for the customer care team division &
		Improve the "outage/service information" process and quality &
		May 1st 2016 \\
		\hline
		% 
		GOAL-4 \label{goal-4}&
		Replace 20\% of the manual process with automated within the customer care team &
		WOMS will be the key to enable future project or changes to automate this process &
		May 1st 2015 \\
		\hline
		%
	\end{longtable}
\end{center}

\subsection{Document Overview and Conventions}
This subsections goal is to provide an overview over the document in hole and to briefly describe what each section is about. For each of the sections a list of affected stakeholders are presented, stakeholders that should read that section. 
Section \ref{sec:introduction} gives an introduction and background to the project. It will also give an overview over the project in whole. This section is suitable for any stakeholder to read, as it is introduction for the whole product and its context.	

Section \ref{sec:overall_description} is a description for the product, its users, dependencies, and constraints. Managers of any department are recommended to read this to get a deeper understanding of the procurement product. Direct users will also benefit from reading 2.3 as it concerns their tasks and environment.

Section \ref{sec:functional_requirements} goal is to list all the requirements connected to the processes in section \ref{sec:overall_description}. This section is mainly written for system developers and users since these requirements are the core in what the system, when developed, will do. 

Section \ref{sec:external_interface_requirements} aims to deliver an explanation on how the system will interact with the current ACME IT-environment. What requirements the environments have on the WOMS. This text is suitable for the IS/IT department and developers.

Section \ref{sec:quality_attributes} is suitable for the IS/IT department and developers. It describes all the non-functional requirements for the new system. 

\subsection{Glossary}
\begin{center}
	\begin{longtable}{|l|l|}
		\caption{Glossary}
		\label{table:glossary}\\
		\hline
		\textbf{Term} & \textbf{Definition}\\
		\hline
		\endfirsthead

		\multicolumn{2}{c}%
		{\tablename\ \thetable\ -- \textit{Continued from previous page}} \\
		\hline
		\textbf{Term} & \textbf{Definition}\\
		\hline
		\endhead

		\hline \multicolumn{2}{r}{\textit{Continued on next page}} \\
		\endfoot

		\hline
		\endlastfoot
		ACC 	& 	The stakeholder \emph{Accounting Team} \\
		\hline
		CC 		& 	The stakeholder \emph{Call Centre Team} \\
		\hline
		CIS 	& 	Customer Information System \\
		\hline
		DMS 	&	Distribution Management System \\
		\hline
		ERP 	& 	Enterprise Resource Planning \\
		\hline
		FT 	& 	The stakeholder \emph{Field Technicians}. Also known as \emph{sub-contractors} \\
		\hline
		IP	&	Internet Protocol  \\ 
		\hline
		LAN &	Local Area Network \\
		\hline
		M/E 	& 	The stakeholder \emph{Maintenance \& Expansion Team} \\
		\hline
		MO 	& 	The stakeholder \emph{Monitoring Team} \\
		\hline
		MoSCoW 	&  A requirement prioritization method \cite{coleyconsulting} \\
		\hline
		NIS 	& 	Network Information System \\
		\hline
		PU 	& 	The stakeholder \emph{Purchasing Team} \\
		\hline
		PDA	& 	Personal Digital Assistant \\
		\hline
		RBP	& 	Related Business Process\\
		\hline
		SQL &	Structured Query Language \\
		\hline
		TLS & 	Transport Layer Security \\
		\hline
		WO 	& 	Work Order \\
		\hline 
	\end{longtable}
\end{center}
